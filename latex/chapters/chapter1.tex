\begin{abstract}
For organic row crops, frequent cultivation is required for weed
control, and these implements require precise guidance systems to
assure proper positioning of the working tools.
Legacy systems have made use of mechanical guiding rods, but these
systems perform poorly in the earliest stages of crop growth.
Modern techniques based on RTK GPS are available commercially but are
prohibitively expensive for small-scale operations.
Therefore, the objective of this study was to develop a low-cost CCD camera system which is capable of supplementing the mechanical row detection during inter-row cultivation.
A computer-vision guidance system was developed for the Intel Atom architecture to interface with an electro-hydraulic steering hitch system.
Two redundant CCD cameras were mounted to the cultivator toolbar
in-line with crop rows to obtain a video stream of the plants passing beneath the implement.
The OpenCV platform was used to develop an algorithm for identifying
the lateral offset of the plant rows and adjust the hydraulic steering
accordingly via PID control.
The computer-vision guidance system was tested successfully without GPS RTK assistance at travel speeds of 6, 8, 10, and 12 km/h in corn and soybean fields under varying ambient light and crop conditions. 
\end{abstract}
\section{Introduction}
\lipsum[1]
\section{Literature Review}
\lipsum[1]
\section{Methodology}
\lipsum[1]
\section{Results}
\lipsum[1]
\section{Discussion}
\lipsum[1]
